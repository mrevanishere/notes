\documentclass[12pt]{book}%
\usepackage{graphicx}%
\usepackage{amsmath}%

\begin{document}

\title{Calculus Notes}
\author{mrevanisworking}

\setcounter{chapter}{4}
\chapter{5 Integrals}
\section{Definition}
Reinmann Sum
\begin{equation}
   \int_a^bf(x)dx=\sum_{i=1}^nf(x_i^*)\Delta x 
\end{equation}
\subsection{Testing Ground}


$\iint_{S} f(x,y) dA $   
\setcounter{chapter}{14}
\chapter{15 Multiple Integrals}
\section{15.1,2,3 Double Integrals }

$\sum_{i=1}^{n}f(x_i^*)\Delta x$\par
Double Integral of $f$ over the rectangle $R$ is 
\begin{equation}
    \int_{R}^{} \int_{}^{}f(x,y)dA- \sum_{i=1}^{m} \sum_{j=1}^{n} f(x_{ij}^*,y_{ij}^*)\Delta A 
\end{equation}
if the limit exists.\par
...
\par
\subsection{Definition of Double Integral}
\begin{equation}
    \int_{R} \int f(x,y)dA=\lim_{(m,n)\to\infty} \sum_{i=1}^{m} \sum_{j=1}^{n} f(x_{ij}^*y_{ij}^*)\Delta A
\end{equation}
iltc double reinmann sum\par
precise definition\par...
 
\subsection{Over Rectangle}
\begin{equation}
    \iint_R f(x,y)dA 
\end{equation}

\subsection{The Midpoint Rule}
\begin{equation}
    \iint_R f(x,y) dA \approx \sum_{i=1}^{m} \sum_{j=1}^{n} f(xover,yover)\Delta A
\end{equation}

\subsection{Fubini's Theorem}
\begin{equation}
    \iint_R f(x,y)dA= \int_{a}^{b} \int_{c}^{d} f(x,y) dydx \int_{c}^{d} \int_{a}^{b} f(x,y) dx dy
\end{equation}
\begin{equation}
    \iint_R g(x)h(y)dA = \int_{a}^{b}g(x)dx \int_{c}^{d}h(y)dy \text{ where } R=[a,b]\times[c,d]
\end{equation}
Average Value of a function
$f_{ave}= \frac{1}{A(R)} \iint_R f(x,y)dA$ 
\subsection{Double Integral over General Region D}
\begin{equation}
    \iint_D f(x,y)dA=\iint_R F(x,y)dA
\end{equation}
Given D then
\begin{equation}
    D= \{(x,y) | a\le x\le b,g_1(x)\le y\le g_x(x)\}
\end{equation}
is
\begin{equation}
    \iint_D f(x,y)dA = \int_{c}^{d} \int_{h_1(y)}^{h_2(2)} f(x,y) dx dy
\end{equation}
and the opposite for y
Subregions can be fixed by solving in terms of a variable
\begin{equation}
    \iint_D f(x,y)dA = \iint_{D_1} f(x,y)dA + \iint_{D_2} f(x,y)dA
\end{equation}
Double Integral of 1 is the Area of D:
\begin{equation}
    \iint_{D} 1dA=A(D)
\end{equation}
\subsection{Double Integrals in Polar Coordinates UNFIN}
Polar Rectangle (disk or circle)
$R=\{(r, \theta )|a\leq r\leq b,\alpha\leq \theta \leq\beta\}$  
where $r^2=x^2+y^2 ,x=r \cos \theta ,y = \sin \theta    $


\section{15.6,7,8 Triple Integrals} 
\subsection{Definition}
Triple integral of $f$  over box $B$  is (if the limit exists)
\begin{equation}
\iiint_B f(x,y,z)dV= \lim_{l,m,n\to \infty} \sum_{i=1}^{l} \sum_{j=1}^{m} \sum_{k=1}^{n} f(x_{ijk}^{*},y_{ijk}^{*},z_{ijk}^{*},)\Delta V
\end{equation}
Fubini's Theorem for Triple Integrals on rect box $B=[a,b]\times[c,d]\times[r,s]$ 
\begin{equation}
    \iiint_B f(x,y,z)dV= \int_{r}^{s} \int_{c}^{d} \int_{a}^{b} f(x,y,z) \,dx\,dy\,dz
\end{equation}
General region E
\begin{equation}
    \iiint_E f(x,y,z)dV= \iiint_B F(x,y,z) dV
\end{equation}
Triple Integral bounded by regions
\begin{equation}
    \iiint_B f(x,y,z)dV= \int_{a}^{b} \int_{g_1(x)}^{g_2(x)} \int_{u_1(x,y)}^{u_2(x,y)} f(x,y,z) \,dz\,dy\,dx
\end{equation}
where terms are switched around when switching orders. Should draw two diagrams. (1) of solid region E and (2) of its projection D onto the xy plane.
\par
3 types of regions. one depending on which dimension goes first
\par
\subsection{Cylindrical Coordinates}
Represented by the ordered triple $(r, \theta ,z)$ where $r$  and $ \theta $  are polar coords to P on x,y and z is distance from P.
\begin{equation}
    x=r \cos \theta ,y=r \sin \theta ,z=z,,,r^2=x^2+y^2, \tan \theta = \frac{y}{x},z=z
\end{equation}
Formula for triple integration in cylindrical coordiantes
\begin{equation}
    \iiint_E f(x,y,z)dV= \int_{\alpha}^{\beta} \int_{h_1(x)}^{h_2(x)} \int_{u_1(r \cos \theta ,r \sin \theta )}^{u_2(r \cos \theta ,r \sin \theta )} f(r \cos \theta ,r \sin \theta ,z) r\,dz\,dr\,d\theta 
\end{equation}

\subsection{Spherical Coordinates}
\begin{equation}
    x=\rho \sin \phi cos \theta, y=\rho \sin \phi \sin \theta ,z=\rho \cos \phi,,,\rho^2=x^2+y^2+z^2
\end{equation}
the counterpart of a rectangular box is a spherical wedge.
\par B is the unit ball.

\section{15.9 Change of Variables in Multiple Integrals } 

Transformation T from the uv-plane to xy-plane: $T(u,v)=(x,y)$
where $x=g(u,v), y=h(u,v)$ 
assuming that T is a $C^1$ transformation (g, h have continuous FOPD) 
\par  
\subsection{Vectors and Jacobian}
Position vector of image
\begin{equation}
    \vec r(u,v) = g(u,v) \hat i+h(u,v) \hat j 
\end{equation}
The Jacobian
\begin{equation}
    \frac{\partial(x,y)}{\partial(u,v)}=\begin{vmatrix}\end{vmatrix}
\end{equation}
\subsection{Change of Variables Theorem}

\section{Applications}
\subsection{}
\subsection{15.4, 15.5 Applications and Surface Area} 
\subsection {15.6 applications of triple integrals}
Special Case where $f(x,y,z)=1$  for all E then the TI is the volume of E
\begin{equation}
    V(E)=\iiint_E dV
\end{equation}
For density function, the triple integral is mass. It's moments are the triple integral times a dimension...physics stuff.
\par There is also probability (joint desnity function)
\begin{equation}
    P((X,Y,Z)\in E)=\iiint_E f(x,y,z)dV
\end{equation}

\end{document}
