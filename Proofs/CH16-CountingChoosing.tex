\documentclass[12pt]{article}
\usepackage{amsmath}
\usepackage{amssymb}
\usepackage{fancyhdr}
%\usepackage{pgfplots}
\usepackage{graphicx}
\usepackage{geometry}
\usepackage{bm}
\usepackage{empheq}

\newcommand{\R}{\mathbb{R}}
\newcommand{\Q}{\mathbb{Q}}
\newcommand{\N}{\mathbb{N}}
\newcommand{\C}{\mathbb{C}}
\newcommand{\Z}{\mathbb{Z}}
\newcommand{\I}{\mathbb{I}}

%allowdisplaybreak
\begin{document}
\title{CH 16 Counting and Choosing}
\author{mrevanishere}
\maketitle


\section{Multiplication Principle}
	THEOREM 16.1 (Multiplication Principle): Let P be a process which 
	consists of n stages, and suppose that for each r, the rth stage
	can be carried out in $ a_r $ ways. Then P can be carried out
	in $ a1-n $ ways.\\
	PROPOSITION 16.1: Let S be a set consisting of n ele. Then the num
	of different arrangements of the elements of S in order is n!
	\begin{align*}
		S &= {a, b, c} \\
		abc, acb, bac, bca, cab, cba
	\end{align*}
\section{Binomial Coefficients}
	DEFINITION: Let n be a positive int and r an int such that 
	$ 0 \le r \le n $ Define
	\begin{align*}
		\begin{pmatrix} n \\ r \end{pmatrix}
	\end{align*}
	("n choose r") to be the num of r-element subsets of $ {1,,,n} $ \\
	PROPOSITION 16.2:
	\begin{align*}
		\begin{pmatrix} n \\ r \end{pmatrix} &= 
		\frac{n!}{r!(n-r)!} 
	\end{align*}
	from proof:
	\begin{align*}
		n! &= \begin{pmatrix} n \\ r \end{pmatrix} \times r! \times (n-r)!
	\end{align*}
	nC0 and nCn = 1 and nC1 = n. \\
	nCr are binomial coefficients from:
\section{Binomial Thoerem}
	THEOREM 16.2 (Binomial Theorem): Let n be a pos int, and let
	a, b be real nums. Then:
	\begin{align*}
		(a + b)^n &= \sum_{r=0}^n \begin{pmatrix} n \\ r \end{pmatrix}
		a^{n-r}b^r \\
		&= a^n + na^{n-1}b + .... see expansion
	\end{align*}
	1. Each expression is symmetrical about the centre:
	\begin{align*}
		\begin{pmatrix} n \\ r \end{pmatrix} &= 
		\begin{pmatrix} n \\ n-r \end{pmatrix}
	\end{align*}
	and from pascal's triangle:
	\begin{align*}
		\begin{pmatrix} n+1 \\ r \end{pmatrix} &= 
		\begin{pmatrix} n \\ r \end{pmatrix} +
		\begin{pmatrix} n \\ r-1 \end{pmatrix}
	\end{align*}
	PROPOSITION 16.3: For any pos int n
	\begin{align*}
		(1 + x)^n &= \sum_{r=0}^n \begin{pmatrix} n \\ r \end{pmatrix} x^r
	\end{align*}
	see 138.
\section{Ordered Selections}
	PROPOSITION 16.4: Let S be a set of ele \\
	(1) The num of ordered selections of r ele of S, allowing repetitions
	is equal to $ n^r $ \\
	(2) The number of ordered selections of r distinct ele of S is equal
	to $ n(n-1) \cdots (n-r+1) $ \\
	Written as $ P(n, r) = \frac{n!}{(n-r)!} = r! 	\begin{pmatrix} n \\ r \end{pmatrix} $ 
\section{Multinomial Coefficients}
	DEFINITION (Ordered Partition): Let n be a pos int, and let
	$ S = {1,,,n} $. A partition of S is a collection of subsets
	S1-k such that each ele of S lies in exactly one of these subsets.
	The partition is ordered if we take account of the order in which the
	subsets are written. \\
	see 140, 141 \\
	PROPOSITION 16.5:
	\begin{align*}
			\begin{pmatrix} n \\ r1-k \end{pmatrix} &= 
			\frac{n!}{r_1!r_2!...r_k!} 
	\end{align*}
	Mult princ: $ n! = 	\begin{pmatrix} n \\ r1-k \end{pmatrix} r_1!r_2!...r_k! $\\
	These are called multinomial coefficients \\
	THEOREM 16.3 (Multinomial Theorem): Let n be a pos int, and let
	x1-k be real numbers. Then the expansion of
	$ (x_1 + \cdots + x_k)^n $ is the sum of all terms of:
	\begin{align*}
			\begin{pmatrix} n \\ r1-k \end{pmatrix} x_1^{r_1}...x_k^{r_k}
	\end{align*}
	where r1-k are nn int such that their sum = n.






\end{document}
