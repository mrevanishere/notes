\documentclass[12pt]{article}
\usepackage{amsmath}
\usepackage{amssymb}
\usepackage{fancyhdr}
%\usepackage{pgfplots}
\usepackage{graphicx}
\usepackage{geometry}
\usepackage{bm}
\usepackage{empheq}

\newcommand{\R}{\mathbb{R}}
\newcommand{\Q}{\mathbb{Q}}
\newcommand{\N}{\mathbb{N}}
\newcommand{\C}{\mathbb{C}}
\newcommand{\Z}{\mathbb{Z}}
\newcommand{\I}{\mathbb{I}}

%allowdisplaybreak
\begin{document}
\title{CH 7 Polynomial Equations}
\author{mrevanishere}
\maketitle


\section{Polynomial Equation}
	is in the form: $ p(x) = 0 $ where p(x) is a polynomial\\
	Degree is the highest power of x with a NZ coefficient \\
	1st degree root: $ -\frac{b}{a} $ \\
	2nd degree root: quadratic formula \\
	3rd degree root:
\section{Solution of Cubic Equations}
	Consider $ x^3  + ax^2 + bx + c = 0 $ \\
	Get rid of $ x^2 $. By $ y = x+\frac{a}{3} $. \\
	Then $ y^3  =  (x +\frac{a}{3)})^3 = $
	$ x^3 + ax^2 + \frac{a^2}{3}x + \frac{a^3}{27} $. \\
	So it becomes $ y^3 + b'y + c' = 0 $ from some b' c' written as:\\
	$ y^3 + 3hy +k =0 $ \\
	Write $ y = u + v $ Then:
	\begin{align*}
		y^3 = (u+v)^3 &=  u^3 + v^3 + 3u^2v + 3uv^2 \\
					  &=  u^3 + v^3 + 3uv(u+v) \\
					  &= u^3 + v^3 + 3uvy
		y^3 - 3uvy - (u^3 + v^3) &= 0
	\end{align*}
	Which has $ u+v $ as a root \\
	Find u, v that: \\
	$ h = -uv, k = -(u^3 + v^3) $ \\
	$ v^3 = -\frac{h^3}{u^3}, u^3 - \frac{h^3}{u^3} = -k $, so
	\begin{align*}
		u^6 + ku^3 -h^3 = 0
	\end{align*}
	This is the quadratic equation
	\begin{align*}
		u^3 =  \frac{1}{2}(-k +\sqrt{^2 + 4h^3})
	\end{align*}
	\begin{align*}
		v^3 =  -k-u^3 = \frac{1}{2}(-k-\sqrt{kk^2+4h^3})
	\end{align*}
	Plug back into $ y= u+v $
	\begin{align*}
	\sqrt[3]{\frac{1}{2}(-k+\sqrt{k^2+4h^3} )} + 
		\sqrt[3]{\frac{1}{2}(-k-\sqrt{k^2+4h^3} )}
	\end{align*}
\section{Higher Degrees}
	There are no formulas for roots of >4th degree equations
\section{The Fundamental Theorem of Algebra}
	Every polynomial equation of a degree at least 1 has a root $ \in\C $\\
	see proof CH 24
\section{Theorem 7.2}
	Every polynomial of degree n factorizes as a product of linear
	polynomials and has exactly n roots $ \in\C $ (counting repeats).
\section{Theorem 7.3 (Complex Conjugates)}
	Every real polynomial factorizes as a product of real linear and real
	quadratic polynomials and has its non-real roots appearing in 
	complex conjugate pairs\\
	see 55/56 for proof
\section{Relationships between Roots, Prop 7.1}
	Let the roots of:
	\begin{align*}
		x^n + a_{n-1}x^{n-1}+\cdots+a_1x+a_0 &= 0
	\end{align*}
	be $ \alpha_1, \alpha_2,\dots, \alpha_n $. If $ s_1 $ denotes the
	sum of the roots, $ s_2 $ denotes the sum of all the products
	of pairs of roots,...
	\begin{align*}
		s_n &= \alpha_1 \alpha_2 \dots \alpha_n \\
			&= (-1)^{n}a_0
	\end{align*}
	see 56/57
\end{document}
