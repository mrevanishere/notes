\documentclass[12pt]{article}
\usepackage{amsmath}
\usepackage{amssymb}
\usepackage{fancyhdr}
%\usepackage{pgfplots}
\usepackage{graphicx}
\usepackage{geometry}
\usepackage{bm}
\usepackage{empheq}

\newcommand{\R}{\mathbb{R}}
\newcommand{\Q}{\mathbb{Q}}
\newcommand{\N}{\mathbb{N}}
\newcommand{\C}{\mathbb{C}}
\newcommand{\Z}{\mathbb{Z}}
\newcommand{\I}{\mathbb{I}}

%allowdisplaybreak
\begin{document}
\title{CH 10}
\author{mrevanishere}
\maketitle


\section{Division}
	Definition:\\
	Let $ a, b\in\Z $. We say a divides b (or a is a factor of b) if 
	$ b = ac $ for some integer c. When a divides b, we write $ a|b $ \\
	Proposition 10.1 \\
	Let a be a positive int. Then for any $ b\in\Z $ there are integers $ q,r $
	such that
	\begin{align*}
		b=qa + r and 0 \le r < a
	\end{align*}
	q is quotient \\
	r is remainder\\
	Proposition 10.2 \\
	Let $ a, b, d\inZ $ and suppose that $ d|a $ and $ d|b $. Then
	$ d|(ma+nb) $for any $ m, n\in\Z $.
\section{The Euclidean Algorithm} 
	Definition:\\
	Let $ a, b\in\Z $. A common factor of a and b is an int that divides both
	a and b. The highest common factor of a and b, written
	$ hcf(a, b) $ is the largest positive integer that divides both a and b.\\
	Algorithm:\\
	Let a, b be ints and let $ d = hcf(a, b) $.\\
	Step 1: Divide a into b and get a quotient and remainder \\
	... HELP \\
	Last nz remainder is the HCF by checking. \\
	General: divide a into b to get q1 and r1; then divide r1 into a,
	getting a remainder r2 < r1; then divide r2 into r1... until 
	rn++1 = 0. Then the rn is the hcf.\\
	Theorem 10.1 \\
	In the above, the highest common factor hcf(a, b) is equal to $ r_n $, 
	the last nz remainder.\\
	Proposition 10.4 \\
	If $ a, b\in\Z $, then any common factor of a and b also divides 
	$ hcf(a, b) $.\\
	Definition: If $ a,b \in\Z $ and $ hcf(a, b) = 1 $, we say that a and b 
	are coprime to each other. \\
	Proposition 10.5: Let $ a, b\in\Z $ \\
	(a) Suppose c is an int such that a, c are coprime to each other,
	and $ c|ab. $. Then $ c|b $ \\
	(b) Suppose p is a prime number and $ p|ab $. Then either $ p|a $ or
	$ p|b $. \\
	Proposition 10.6 \\
	Let a1-n $ \in\Z $ and let p be a prime number. If $ p|a1-n $ then
	$ p|a_i $ for some i.






\end{document}
