\documentclass[12pt]{article}
\usepackage{amsmath}
\usepackage{amssymb}
\usepackage{fancyhdr}
%\usepackage{pgfplots}
\usepackage{graphicx}
\usepackage{geometry}
\usepackage{bm}
\usepackage{empheq}

\newcommand{\R}{\mathbb{R}}
\newcommand{\Q}{\mathbb{Q}}
\newcommand{\N}{\mathbb{N}}
\newcommand{\C}{\mathbb{C}}
\newcommand{\Z}{\mathbb{Z}}
\newcommand{\I}{\mathbb{I}}

%allowdisplaybreak
\begin{document}
\title{CH 6 Complex Numbers}
\author{mrevanishere}
\maketitle


\section{Intro}
	i is $ i^2 = -1 $.\\
	A complex number is $ z = a+bi $,
	where the real part is $ a= Re(z) $ and the imaginary part is
	$ b= Im(z) $\\
	Addition and multiplication is defined for complex nums (39)
	\begin{align*}
		\frac{1-i}{1+i} &= -i 
	\end{align*}
	$ \C $ is the set of all complex numbers \\
	$ \R \subseteq C $ \\
	for quadratics:\\
	if $ b^2 \ge 4ac $ then $ \in\R $,\\
	if $ b^2 \le 4ac $ then $ \in\C $
\section{Geometry}
	Complex conjugate of $ z= a+bi $ is defined as:\\
	$ \overline{z} = a-bi $ (reflected over real axis) \\
	Modulus of z is distance from the origin to z (|z|)
	$ z \overline{z} = |z|^2 $ \\
	Argument of z is the angle $ \theta $ \\
	Polar form of z: $ z= r(\cos\theta +i\sin\theta) $ \\
	where $ a= r\cos\theta $ and $ b= r\sin\theta $ and $ |z| = r $\\
	Principal argument is in $ -\pi < \theta \le \pi $ written as 
	$ arg(z) $, because multiples of 2pi are same angle\\
	Ex $ arg(1-i) = -\frac{\pi}{4} $
\section{De Moivre's Theorem}
	Complex plane or argand diagram\\
	If $ z_1 = r_1(\cos\theta_1 + i\sin\theta_1) $ and
	If $ z_2 = r_2(\cos\theta_2 + i\sin\theta_2) $
	Then $ z_1z_2 $ has mmodulus $ r_1r_2 $ and arugment $ \theta_1\theta_2 $\\
    Says: mult a C z by $ \cos\theta + i\sin\theta $ ROTATES z 
	counterclockwise through the angle $ \theta $ \\
	Ex: multipliciation by i rotates $ \frac{pi}{2} $
	\subsection{Prop 6.1}
		Let $ z= r(\cos\theta+i\sin\theta) $ and let n be a pos int:
		\begin{align*}
			(i) z^n &=  r^n(\cos n\theta +i\sin n\theta) \\
			(ii) z^{-n} &= r^{-n}(\cos n\theta - i\sin n\theta)
		\end{align*}
\section{The $ e^{i\theta} $ Notation }
	$ e^{i\theta} = \cos\theta + i\sin\theta $ \\
	Common:
	\begin{align*}
		e^{2\pi i} &= 1 \\
		e^{\pi i} &= -1 \\
		e^{\frac{pi}{2} i} &=  i \\
		e^{\frac{pi}{4} i} &= \frac{1}{\sqrt{2}}(1+i)
		e^{i\theta} &=  e^{i(\theta +2k\pi)}
	\end{align*}
	all $ e^{i\theta} $ has modulus 1 (unit circle) \\
	$ z= re^{i\theta} $ where $ r = |z| $ and $ \theta = arg(z) $
	\begin{align*}
		e^{i\theta}e^{i\phi} &= e^{i(\theta + \phi)}	\\
		(e^{i\theta})n &= e^{in\theta}
	\end{align*}
\section{Roots of Unity}
	cube roots of unity are each $ \frac{2\pi}{3} $ away from eachother \\
	nth Roots of unity: if n is a pos int, then the complex nums that 
	satisfy the equation are:
	\begin{align*}
		z^{n} &= 1
	\end{align*}
	\subsection{Prop 6.3}
		Let n be a pos int and $ w := e^{\frac{2\pi i}{n} } $ Then the
		nth roots of unity are the n complex nums:
		\begin{align*}
			1,w,w^2,...,w^{n-1}
		\end{align*}
		and are evenly spaced around the unit circle
\end{document}
