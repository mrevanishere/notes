\documentclass[12pt]{article}
\usepackage{amsmath}
\usepackage{amssymb}
\usepackage{fancyhdr}
%\usepackage{pgfplots}
\usepackage{graphicx}
\usepackage{geometry}
\usepackage{bm}
\usepackage{empheq}

\newcommand{\R}{\mathbb{R}}
\newcommand{\Q}{\mathbb{Q}}
\newcommand{\N}{\mathbb{N}}
\newcommand{\C}{\mathbb{C}}
\newcommand{\Z}{\mathbb{Z}}
\newcommand{\I}{\mathbb{I}}

%allowdisplaybreak
\begin{document}
\title{CH 18 Equivalence Relations}
\author{mrevanishere}
\maketitle

\section{Relation}
	RELATION: Let S be a set, a relation on S is: \\
	Choose a subset R of the Cartesian product $ S \times S $, or
	R consist of some of the ordered pairs $ (s, t), s, t \in S $. For those
	ordered pairs $ (s, t) \in R $, we write $ s ~ t $ and say
	s is related to t. $ (s, t ) \not\in R, s \nsim t $, is a relation on S \\
	EQUIVALENCE RELATION: Let S be a set, and let ~ be a relation on S. Then
	~ is an equivalence relation if for all $ a, b, c \in S $:
	\begin{align*}
		(i)& a ~ a \text{ this says ~ is reflexive } \\
		(ii)& \text{ if } a ~ b \text{ then } b ~ a ( this says ~ is symmetric )\\
		(iii)& \text{ if } a ~ b \text{ and } b ~ c \text{ then } a~c \text{ says ~ is transitive }
	\end{align*}
\section{Equivalence Classes}
	Let S be a set and ~ an equivalence relation on S. For $ a \in S $, define
	\begin{align*}
		cl(a) &= \{ s|s \in S, s ~ a \}
	\end{align*}
	Thus, $ cl(a) $ is the set of things that are related to a. The subset
	$ cl(a) $ is called an equivalence class of ~. The EC of ~ are the subsets
	$ cl(a) $ as a ranges over the ele of S.
\section{Prop 18.1}
	Let S be a set and let ~ be an ER on S. Then the EC of ~ form a 
	partition of S.




\end{document}
