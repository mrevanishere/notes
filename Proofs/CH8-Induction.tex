\documentclass[12pt]{article}
\usepackage{amsmath}
\usepackage{amssymb}
\usepackage{fancyhdr}
%\usepackage{pgfplots}
\usepackage{graphicx}
\usepackage{geometry}
\usepackage{bm}
\usepackage{empheq}

\newcommand{\R}{\mathbb{R}}
\newcommand{\Q}{\mathbb{Q}}
\newcommand{\N}{\mathbb{N}}
\newcommand{\C}{\mathbb{C}}
\newcommand{\Z}{\mathbb{Z}}
\newcommand{\I}{\mathbb{I}}

%allowdisplaybreak
\begin{document}
\title{CH 8 - Induction}
\author{mrevanishere}
\maketitle


\section{Principle of Mathematical Induction}
	Suppose that for each positive integer n we have a statement $ P(n) $.
	If we prove the following two things: \\
	$ (a) P(1)$ is true;\\
	$ (b) $ for all n, if $ P(n) $ is true then $ P(n+1) $ is also true;\\
	then $ P(n) $ is true for all positive integers n 
\section{Principle of Mathematical Induction 2}
	Let k be an integer. Suppose that for each integer $ n \ge k $ we have a 
	statement $ P(n) $. If we prove that following two things:\\
	(a) $ P(k) $ is true; \\
	(b) for all $ n \ge k $, if $ P(n) $ is true then $ P(n+1) $ is also true;\\
	then $ P(n) $ is true for all integers $ n \ge k $.
\section{Factorial}
	n factorial is defined as:\\
	$ n! = n(n-1)(n-2)...3\cdot 2\cdot 1 $ \\
	$ 0! = 1 $
\section{Guesswork}
	Some problems have to use guesswork to identify a pattern first.
\section{Summation Notation}
	If f1-n are numbers we abbreviate the sum of all of them by
	\begin{align*}
		f1 + ...+fn &= \sum^{n}_{r=1}f_r
	\end{align*}
	Some summation algebra:
	\begin{align*}
		\sum^n_{r=1}(af_r + bg_r +c) &= 
		a\sum^n_{r=1}f_r +b\sum^n_{r=1}g_r + cn
	\end{align*}
\section{Geometric Examples}
	see later
\section{Prime Factorization}
	Definition: a prime number is a positive integer p such that $ p \ge 2 $
	and the only positive integers diving p are 1 and p.\\
	Proposition 8.1 (Prime facorization):\\
	Every positive int greater than 1 is equal to a product of prime numbers.
\section{Principle of Strong Mathematical Induction}
	Suppose that for each integer $ n\ge k $ we have a statement $ P(n) $.
	If we prove the following two things: \\
	(a) $ P(k) $ is true;\\
	(b) for all n, if $ P(k), P(k+1),...,P(n) $ are all true, then $ P(n+1) $
	is also true \\
	then P(n) is true for all $ n\ge k $.
\section{Cauchy's Inequality: Proposition 8.2}
	Let n be a positive integer. Then for any real numbers a1-n and b1-n:
	\begin{align*}
		sum a_1b_1 to a_nb_n &\le \sqrt{a_1^2+...+a_n^2} \sqrt{b_1^2+...+b_n^2} 
	\end{align*}

	


\end{document}
