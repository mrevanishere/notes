\documentclass[12pt]{article}
\usepackage{amsmath}
\usepackage{amssymb}
\usepackage{fancyhdr}
%\usepackage{pgfplots}
\usepackage{graphicx}
\usepackage{geometry}
\usepackage{bm}
\usepackage{empheq}

\newcommand{\R}{\mathbb{R}}
\newcommand{\Q}{\mathbb{Q}}
\newcommand{\N}{\mathbb{N}}
\newcommand{\C}{\mathbb{C}}
\newcommand{\Z}{\mathbb{Z}}
\newcommand{\I}{\mathbb{I}}

%allowdisplaybreak
\begin{document}
\title{CH 4 nth Roots and Rational Powers}
\author{mrevanishere}
\maketitle

\section{Prop 4.1}
	Let n be a positive integer. If x is a positive real num, then there is
	exactly one positive real number y such that $ y^n = x $
	\begin{align*}
		y = x^{\frac{1}{n}}
	\end{align*}
	Rules: \\
	$ x > 0 $,integer m $ x^m (m\in \Z) $: if $ m < 0 $ 1/$ x^m $, m=0, =1 \\
	Let $ \frac{m}{n} \in \Q (m, n \in \Z and n \ge 1)$ Then:
	\begin{align*}
		x^{\frac{m}{n}} = (x^{\frac{1}{n}}^m)
	\end{align*}
\section{Prop 4.2, rules of exponents}
	Let x, y be positive real num and p,q $ \in\Q $, Then:
	\begin{align*}
		(i) x^p x^q &= x^{p+q} \\
		(ii) (x^p)^q &=  x^{pq} \\
		(iii) (xy)^p = x^p y^p 
	\end{align*}
	see pg 28 for proofs
	




\end{document}
