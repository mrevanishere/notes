\documentclass[12pt]{article}
\usepackage{amsmath}
\usepackage{amssymb}
\usepackage{fancyhdr}
%\usepackage{pgfplots}
\usepackage{graphicx}
\usepackage{geometry}
\usepackage{bm}
\usepackage{empheq}

\newcommand{\R}{\mathbb{R}}
\newcommand{\Q}{\mathbb{Q}}
\newcommand{\N}{\mathbb{N}}
\newcommand{\C}{\mathbb{C}}
\newcommand{\Z}{\mathbb{Z}}
\newcommand{\I}{\mathbb{I}}

%allowdisplaybreak
\begin{document}
\title{CH 13 Congruence of Integers}
\author{mrevanishere}
\maketitle

\section{Congruence}
	DEFINITION: Let m be a pos int. For $ a, b\in\Z $ if m divides
	$ b - a $ we write $ a \equiv b \text{ mod } m $ and say a is
	congruent to b modulo m \\
	Prop 13.1: Every int is congruent to exactly one of the numbers
	0 to m-1 modulo m \\
	Prop 13.2: Let m be a pos int. The following are true,
	for all $ a, b, c, \in\Z $: \\
	(1) $ a \equiv a \mod m $ \\
	(2) if $ a \equiv b \mod m $ then $ b \equiv a \mod m $ \\
	(3) if $ a \equiv b \mod m $ and $ b \equiv c \mod m $, then
	$ a \equiv c \mod m $
\section{Arithmetic with Congruences}
	Prop 13.3: Suppose $ a \equiv b \mod m $ and $ c \equiv d \mod m $. Then:\\
	$ a + c \equiv b + d \mod m $ and $ ac \equiv bd \mod m$ . \\
	Prop 13.4: If $ a \equiv b \mod m $, and n is a pos int, then \\
	$ a^n \equiv b^n \mod m $ \\
	Prop 13.5: \\
	(1) Let a and m be coprime integers. If $ x, y \in\Z $ are such that
	$ xa \equiv ya \mod m $, then $ x \equiv y \mod m $ \\
	(2) Let p be a prime, and let a be an int that is not divisible by p.
	If $ x, y \in\Z $ are such that $ xa \equiv ya \mod p $, then
	$ x \equiv y \mod p $
\section{Congruence Equations}
	$ ax \equiv b \mod m, x\in\Z $ is a linear congruence equation \\
	Prop 13.6: The congruence equation $ ax \equiv b \mod m $ has a 
	solution $ x\in\Z $ iff $ hcf(a, m) $ divides b

\section{The System $ \Z_m $}
	"the integers modulo m". \\
	Examples

\end{document}
