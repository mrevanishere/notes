\documentclass[12pt]{article}
\usepackage{amsmath}
\usepackage{amssymb}
\usepackage{fancyhdr}
%\usepackage{pgfplots}
\usepackage{graphicx}
\usepackage{geometry}
\usepackage{bm}
\usepackage{empheq}

\newcommand{\R}{\mathbb{R}}
\newcommand{\Q}{\mathbb{Q}}
\newcommand{\N}{\mathbb{N}}
\newcommand{\C}{\mathbb{C}}
\newcommand{\Z}{\mathbb{Z}}
\newcommand{\I}{\mathbb{I}}

%allowdisplaybreak
\begin{document}
\title{CH 5 Inequalities}
\author{mrevanishere}
\maketitle

inequality: statement about real num (bool)
\section{Rules 5.1 of inequalities}
	(1) If $ x\in\R $ then either $ x>0 $ or $ x<0 $ or $ x=0 $ (only one is true)\\
	(2) If $ x>y $then$ -x<-y $\\
	(3) If $ x>y $ and $ c\in\R $, then $ x+c>y+c $ \\
	(4) If $ x>0 $and$ y>0 $, then$ xy>0 $\\
	(5) If $ x>y $and$ y>z $then$ x>z $ \\
	rule 3 implies rule 2 \\
	see pg 32 proofs

	important:xu > xv proof
	modulus || as abs \\
	See: Arithmetic-Geometric Mean Inequality: \\
	if n is a positive integer and a1-n are positive reals, then
		\begin{align*}
			()a_1a_2...a_n)^{\frac{1}{n}}\le \frac{1}{n}(a_1+a_2+...a_n)
		\end{align*}
	LHS is geometric mean, RHS is arithmetic mean




\end{document}
