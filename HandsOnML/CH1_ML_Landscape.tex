\documentclass[12pt]{article}
\usepackage{amsmath}
\usepackage{amssymb}
\usepackage{fancyhdr}
%\usepackage{pgfplots}
\usepackage{graphicx}
\usepackage{geometry}
\usepackage{bm}
%allowdisplaybreak
\begin{document}
\title{CH 2 End-to-End ML Project}
\author{mrevanisworking}
\maketitle

1. Look at big picture \\
2. Get the data \\
3. Discover and visualize the data to gain insights \\
4. Prepare the data for ML algorithms \\
5. Select a model and train it \\
6. Fine-tune your model \\
7. Present your solution \\
8. Launch, monitor, and maintain your system
\subsection{Working with Real Data}
	\subsubsection{Popular Places to get datasets}
		UC irvine ML Repo \\
		Kaggle, Amazon AWS, Data Portals, Quandl, Wikipedia, Quora, reddit
	\subsubsection{This chapter's dataset}
		California Housing Prices (1990 California census)
\subsection{Look at the Big Picture}
	Task: use California census data to build a model of housing prices
	in the state. \\
	Data includes: population, median income, median housing price
	for each block group\\
	Block group: the smallest geographical unit for which the UC Census
	Bureau publishes sample data "distrcts"
	\subsubsection{Frame the Problem}
		Ask what the business objective is \\
		How does the company expect to use and benefit from this model?\\
		This Answer: model's output will be fed to another ML system.\\
		Signal: a piece of info fed to a ML sys \\
		Ask what the current solution is (if any) \\
		Frame the problem:\\
		is it supervised, unsupervised, RL (Reinforcement Learning) ?\\
		is it classification, regression, or something else?\\
		batch learning or online learning?\\
		Answer: since given labeled training examples, it is a supervised learning
		task. regression because predict value. (multiple regression, 
		univariate regression)
	\subsubsection{Select a Performance Measure}
		a typical performance measure for regression is RMSE

	\subsubsection{Pipeline}
		a secuence of data processing components. components usually run
		ASYNC.





\end{document}
