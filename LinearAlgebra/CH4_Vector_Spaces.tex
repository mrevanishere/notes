\documentclass[12pt]{article}
\usepackage{amsmath}
\usepackage{amssymb}
\usepackage{fancyhdr}
%\usepackage{pgfplots}
\usepackage{graphicx}
\usepackage{geometry}
\usepackage{bm}


\newcommand{\R}{\mathbb{R}}

%\allowdisplaybreak
\begin{document}
\title{CH 3 Determinants}
\author{mrevanisworking}
\maketitle

\subsection{Vector Spaces and Subspaces}
    \subsubsection{Vector Space Definition}
        A nonempty set V of vectors, where addition and scalar mult
        are defined by the 10 commandments and 3 more. SEE 192,193
    \subsubsection{Subspaces}
        See CH2 for DEF \\
        subspace is guaranteed a vector space \\
        subspace is used when at least 2 vector spaces are in mind
    \subsubsection{Vector Space, Span }
        Linear Combination: any sum of scalar multiples of vectors,
        Span is set of all vec that can be written as LC \\
        if vectors are in a VS, then Span of vectors is a subspace
        of V.
\subsection{Null Spaces, Column Spacess, and Linear Transformations}
    \subsubsection{Null Space of Matrix}
        see CH2 notes \\
        set of all $\bm{x}\in\R^n$  mapped to zero vector via LT
    \subsubsection{Null Space is Subspace Theorem}
        NS is subS of $\R^n$.
        Set of all Solutions to $A\bm{x}= \bm{0}$  of m HLE in n 
        unknowns is a subS of $\R^n$ 
    \subsubsection{Explicit Description Nul A}
        solving $A\bm{x}= \bm{0}$  produces explicit description of
        Nul A \\
        SEE EXAMPLE 3
    \subsubsection{Col Space of MX}
    $\text{Col} A = \text{Span}\{\bm{a_{1}},\dots ,\bm{a_{n}}\}$ \\
        Col space is a subS of $\R^n$
        \begin{align*}
            \text{Col} A = \{\bm{b}:\bm{b} = A\bm{x} \text{for some}\bm{x}
            \in\R^n\}
        \end{align*}
        Col space is range of LT\\
        CS of mxn A is all of $\R^m$  IFF the equation
        $A\bm{x} = \bm{b}$  has a sol for each $\bm{b}\in\R^m$ 
    \subsubsection{Contrast between Nul A and Col A}
        when MX not square, Nul A and Col A are separate.\\
        when square, Nul A and Col A share ZV, and some others in special cases\\
        SEE 206 
    \subsubsection{Kernel and Range of LT}
        a LT from VS into another VS is a rule that assigns 
        each $\bm{x}$  to a unique number $T(\bm{x})$ such that
        vector addition and scalar mult is defined.\\
        Kernel (null space of LT) \\
        Range (col space of LT) \\
        see 3Blue1Brown \\
        DIFFERENTIATION IS A LINEAR TRANSFORMATION see ex7 
\subsection{Linear Independent Sets; Bases}
    \subsubsection{THEOREM 4: sec 1.7}
    \subsubsection{Definition of Basis see previous ch}
    \subsubsection{Spanning Set Theorem}
        S is a set of vectors in V, H is span of that set of vectors \\
        a. if one vec in S is a LC of the remaining, then the set
        formed by S removing that vec still spans H \\
        b. if $H\ne \{\bm{0}\}$, some subset of S is a basis of H \\
        basis is a spanning set small as possible to be linearly independent
    \subsubsection{Bases for Nul A and Col A}
        base for Nul A is in 4.2 LIS example.\\
        base of Col A SEE EX 8. NPC is an LC of the PC. \\
        FACT: PC of A form a basis for Col A. (EFM)
\subsection{Coordinate Systems}
    \subsubsection{The Unique Representation Theorem}
        let B be the basis, then for each $\bm{x}\in V$  there
        exists a unique set of scalars c that
        \begin{align*}
            \bm{x} = c_{1}\bm{b_{1}}+ \cdots + c_{n}\bm{b_{n}} 
        \end{align*}
        (Just think of basis vectors spanning around space)
        \subsubsection{Definition of Coordinates (see CH2)}
            coordinate mapping by B
        \subsubsection{Change of Coordinates Matrix}
            \begin{align*}
                P_{\mathcal{B}} = \begin{bmatrix}
                \bm{b_{1}} & \cdots & \bm{b_{n}}
                \end{bmatrix}
            \end{align*}
            and $ \bm{x} = \cdots+ c_{n}\bm{b_{n}} $  
            \begin{align*}
                \bm{x} = P_{\mathcal{B}} \big[\bm{x}\big]_{\mathcal{B}}
            \end{align*}
            where $P_{\mathcal{B}}$ is a change-of-coordinates MX from B
            to standard basis in $\R^n$ 
        \subsubsection{Coordinate mapping is 1-1 LT Theorem 8}
            B is basis for VS V, then the coordinate mapping
            \begin{align*}
                \bm{x} \mapsto \big[\bm{x}\big]_{\mathcal{B}}
            \end{align*}
            is a 1-1 LT from V onto $\R^n$ 
\subsection{Dimension of a VS}
    \subsubsection{>n Basis set is Linearly Dependent Theorem}
        if basis B has n vectors, then any set in V that has more 
        than n vectors is LD.
    \subsubsection{All bases of a VS Theorem}
        If a VS V has a basis of n vec, then every basis of V 
        must have n vec
    \subsubsection{Definition of Dimension}
        if V is spanned by a finite set, then V is finite-dimensional,
        and dim V is the num of vec in basis for V.
        dim of 0 VS is zero. If V is not spanned by finite set,
        V is infinite-dimensional.
    \subsubsection{Subspaces of Finite-Dimensional Space Theorem(FDS)}
        SUBSPACE THEOREM 11: let H be subS of FDVS V.
        any LI set in H can be expanded to a basis for H,
        H is FD and dim $H\le \text{dim} V$ 
    \subsubsection{The Basis Theorem}
        see CH 2
    \subsubsection{Dimensions of Nul A and Col A}
        dim Nul A is num of FV in $A\bm{x}= \bm{0}$  \\
        dim Col A is num of PC in A
\subsection{Rank}
    \subsubsection{Row Space}
        set of all LC of row vec is row space RS of A,
        denoted by Row A. Row A is a subS of $\R^n$ 
    \subsubsection{Row Space Theorem}
        If two MX are RE, then RS is the same.
        If B is in EFM, the NZ rows of B form a basis for the RS of
        A as well as B.
    \subsubsection{Rank Theorem see CH2}
    
        SEE EX5
    \subsubsection{Rank, Invertible MX Theorem}
        all true/all false:\\
        m. The columns of A form a basis of $\R^n$ \\
        n.  $\text{Col}A = \R^n $ \\  
        o. $ \text{dim Col}A = n $  \\
        p. $ \text{rank} A = n$  \\
        q. $ \text{Nul}A = \{\bm{0}\} $ \\
        r. $ \text{dim Nul} A = 0 $ 
        pg 254 for Row space of IMX Theorem
\subsection{Change of Basis}
    \subsubsection{Change of Coordinates MX from B to C Theorem}
        B is b basis and C is c basis of V, then there is unique nxn MX
        such that
        \begin{align*}
            \big[\bm{x}\big]_{\mathcal{C}} =
            P_{\mathcal{C}\leftarrow \mathcal{B}}  \big[\bm{x}\big]_{\mathcal{B}}
        \end{align*}
        SEE FIG2 \\
        inverse is b larrow c
    \subsubsection{Change of Basis in $\R^n$ }
        B to the standard basis E is still B \\
        SEE 243 for formula and notation
        $ P_C^{-1}P_B = P_{C\leftarrow B} $ 
\subsection{APPLICATIONS to Difference Equations}
    later
\subsection{APPLICATIONS to Markov Chains}
    learn in stats

\end{document}
