\documentclass[12pt]{article}
\usepackage{amsmath}
\usepackage{amssymb}
\usepackage{fancyhdr}
%\usepackage{pgfplots}
\usepackage{graphicx}
\usepackage{geometry}
\usepackage{bm}


\newcommand{\R}{\mathbb{R}}
\newcommand{\C}{\mathbb{C}}

%\allowdisplaybreak
\begin{document}
\title{CH 5 Eigenvectors and Eigenvalues}
\author{mrevanisworking}
\maketitle

\subsection{Eiegenvectors and Eigenvalues}
    recheck EV and Evec.
    ****************
    \subsubsection{Definition}
        Eigenvector (Evec) of nxn A is NZ vec $\bm{}$ such that
        $A\bm{x} = \lambda \bm{x} $  for some $\lambda$. \\
        Eigenvalue (EV) is that scalar $\lambda $ of A if there is a 
        nontrivial solution $\bm{x}$  of the equation. or the
        equation 
        \begin{align*}
            (A-\lambda I) \bm{x} = \bm{0}
        \end{align*}\\
        $\bm{x}$  is the Evec corresponding to $\lambda$  
    \subsubsection{Eigenspace}
       set of all solutions of that equation is the nullspace of
       $A-\lambda I$. The set is a subS of $\R^n$  called the 
       eigenspace (ES) of A corresponding to $\lambda $.\\
       has ZV and all EV of $\lambda $ \\
       See FIG3 (A acts on ES as dilation)
    \subsubsection{EV of Triangular Theorem}
        EV of triMX are the entries on its main diag. \\
        0 is an EV of A IFF A is not invertible.
    \subsubsection{Linearly Independent EV Theorem}
        if v are EV corresponding to $\lambda $  of nxn A,
        then the set v is LI.
    \subsubsection{EV and Difference Equations}
        SINCE APPLICATION later
\subsection{The Characteristic Equation}
    \subsubsection{Determinants SEE CH3}
    \subsubsection{The Characteristic Equation}
        a scalar is an EV of nxn A IFF $\bm{\lambda }$  satisfies
        $\text{det}(A - \lambda I) = 0$ \\
        solving gives a characteristic polynomial of A.\\
        Some EV have multiplicities. \\
        Complex roots have complex EV (CEV).
    \subsubsection{Similarity Theorem}
        if nxn A,B are similar, then they have same
        characteristic polynomial and the same EV with same
        multiplicites.\\
        1. see warning \\
        2. Similarity $\ne $  RE, RO usually changes EV.
    \subsubsection{APPLICATIONS to Dynamical Systems}
        later
\subsection{Diagonalization}
    EV-Evec info can be displayed as $A = PDP^{-1}$  where D is DMX.
    SMX is diagonalizable if A is similar to a DMX.
    \subsubsection{The Diagonalization Theorem}
        nxn A is diagonalizable IFF A has n LD Evec.
        $A = PDP^{-1}$  where D is DMX, IFF the cols of P are n
        LI Evec of A. diag entries of D are Evec of A that correspond
        to the Evec in P. \\
        MEANING: A is diagonalizable IFF enough Evec to form 
        basis of $\R^n$  which is called the eigen vector basis of
        $\R^n$. (EvecB) 
    \subsubsection{Diagonalizing Matricies}
        Find an IMX P, and DMX D  to satisfy equation in Diagonalization
        Theorem.\\
        1. Find the EV of A (characteristic equation) \\
        2. Find n linearly independent Evec of A \\
        3. Construct P from the vec in step 2\\
        4. Construct D from the corresponding EV.
    \subsubsection{Diagonalizable Theorem}
        nxn MX w/ n distinct EV is diagonalizable.
    \subsubsection{Not Distinct EV Diagonalizable Theorem}
        a. for $1\le k\le p$  the dim of ES for $\lambda _{k}$  is less
        than or equal the multiplicity of the EV $\lambda _{k}$  \\
        b. A is diagonalizable IFF the sum of dim of ES equals n,
        that happens IFF:
        (i) characteristic polynomial factors into linear factors,
        (ii) the dim of ES for each $\lambda _{k}$  equals the multiplicity
        of $\lambda _k$  \\
        c. if A is diagonalizable and $\mathcal{B_{k}}$  is a basis for the
        ES wrt $\lambda_k$  for each k, then total collection of vec
        in sets B 1-p forms an Evec basis for $\R^n$\\
        SEE EX6
\subsection{Eigenvectors and Linear Transformations}
    \subsubsection{MX of LT}
        \begin{align*}
            [T(\bm{x})]_C = M[\bm{x}]_B
        \end{align*}
        Mis called the MX for T relative to the bases B and C \\
        See FIG1, FIG2, EX 1
    \subsubsection{Linear Transformations from V into V}
        HELP, SEE EX2
    \subsubsection{Diagonal Matrix Representation Theorem}
        Suppose APDP (from above) where D is a diagonal nxx. If B
        is basis for $\R^n$  formed from col of P, then D is
        the B-MX for the transformation
        \begin{align*}
            \bm{x} \mapsto A\bm{x}
        \end{align*}
        MEANING: describing same LT with different bases
    \subsubsection{Similarity of MX Representations}
        the set of all MX similar to A coincides with the
        set of all MX representations of the transformation
        \begin{align*}
            \bm{x} \mapsto A\bm{x}
        \end{align*}
\subsection{Complex Eigenvalues}
    \subsubsection{See Examples}
        complex Evec describe rotations.\\
        See FIG1
    \subsubsection{Real and Imaginary Parts of Vectors}
        $\text{Re}\bm{x}$  is real part of complex conjugate,\\
        $ \text{Im}\bm{x} $ is imaginary part of complex conjugate,\\
        Properties of complex conjugates carry over to complex
        matrix algebra. see 300.
    \subsubsection{Evec and EV of Real MX that acts on $\C^n$ }
        \begin{align*}
            A\bm{\overline{x}} = \overline{A\bm{x}} = 
            \overline{\lambda \bm{x}} = \overline{\lambda }\overline{\bm{x}}
        \end{align*}
        when A is real, its CEV occur in conjugate pairs.\\
        See EX 6/FIG2/FIG3:\\
        angle $\phi $  is the argument of $\lambda= a + bi$  .\\
        transformation  $\bm{x}\mapsto C\bm{x}$  can be viewed as the
        composition of a rotation through $\phi $  and scaling by 
        $|\lambda |$ 
    \subsubsection{APCP Theorem}
        let A be real 2x2 w/ CEvec $\lambda = a-bi(b\ne 0)$  and
        an associated Evec $\bm{v}$  in $\C^2$, then:  
        \begin{align*}
            A=PCP^{-1}\,  \text{where} P = 
            [\text{Re}\bm{v}\,\text{Im}\bm{v}] \text{and}
            C= \begin{bmatrix}
            a&-b\\
            b&a\\
            \end{bmatrix}
        \end{align*}
        a plane is invariant under A if it is a rotated plane.
\subsection{APPLICATIONS Discrete Dynamical Systems}
    later in round 2
\subsection{APPLICATIONS to Differential Equations}
    later in differential equations
\subsection{Iterative Estimated Eigenvalues}
    \subsubsection{The Power Method}
    \subsubsection{The Inverse Power Method}
        
\end{document}
