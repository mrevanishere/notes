\documentclass[12pt]{article}
\usepackage{amsmath}
\usepackage{amssymb}
\usepackage{fancyhdr}
%\usepackage{pgfplots}
\usepackage{graphicx}
\usepackage{geometry}
\usepackage{bm}


\newcommand{\R}{\mathbb{R}}
\newcommand{\C}{\mathbb{C}}

%\allowdisplaybreak
\begin{document}
\title{CH 7 Symmetric Matricies and Quadratic Forms}
\author{mrevanisworking}
\maketitle

\subsection{Diagonalization of Symmetric Matricies}
    Symmetric MX: a MX that $A^T = A$ (is square)
    maybe: row i = col i?
    \subsubsection{Orthogonal Evec Symmetry Theorem}
        if A is symmetric, then any two Evec from different
        ES are orthogonal.
    \subsubsection{Orthogonally Diagonalizable Theorem}
        a nxn is orthogonally diagonalizable if there are
        an orthogonal P ($P^{-1}= P^T$ ) and d such that:
        \begin{align*}
            A= PDP^T = PDP^{-1}
        \end{align*}
        Theorem: nxn A is orthogonally diagonalizable IFF
        A is a symmetric matrix.
    \subsubsection{Spectral Theorem}
        set of Evec of A is sometimes called the spectreum of A.\\
        Theorem: a nxn symmetric A has:\\
        a. A has n real Evec, counting multiplicities.\\
        b. dim of ES for each EV $\lambda $  equals the multiplicity
        of $\lambda $  as a root of the chara equation.\\
        c. ES are mutually orthogonal: Evec corresponding to different
        EV are orthogonal.\\
        d. A is orthogonally diagonalizable.
    \subsubsection{Spectral Decomposition}
        \begin{align*}
            A = \cdots + \lambda _{n}\bm{u_{n}}\bm{u_{n}^{T}}
        \end{align*}
        because it breaks up A into pieces determined by the 
        spectrum of A.\\
        $\bm{u_{j}}\bm{u_{j}}^T$  is a projection matrix:\\
        for each x, the vector $()\bm{u_{j}}\bm{u_{j}}^T)\bm{x}$ is the
        orthogonal projection of $\bm{x}$  onto subS spanned by $\bm{u_{j}}$  
\subsection{Quadratic Forms}
    a quadratic form on $\R^n$  is a fxn Q defined on  $\R^n$  whose
    value at $\bm{x}$  is $\R^n$  can be computed by 
    $ Q(\bm{x}) \bm{x}^TA\bm{x} where A is nxn symmetric. $ \\
    That A is the MX of the quadratic form.\\
    quadratic forms are easier when no cross-product terms (when 
    MX of quadratic form is diagonal)\\
    cross-product term can be eliminated with change of variables
    \subsubsection{Change of Variable in a Quadratic Form}
        if $\bm{x}$  is a variable vec in rn, then change of variable
        is an equation of the form:
        \begin{align*}
            \bm{x} = P\bm{y}\,\,\,\text{or}
            \,\,\, \bm{y}= P^{-1}\bm{x}
        \end{align*}
        CHANGE OF VARIABLES (not change of coordinates)\\
        *read 404 and check Theorem 2\\
        SEE FIG1
    \subsubsection{The Principal Axes Theorem}
        A is nxn symmetric, then there is an orthogonal change of
        variable $\bm{x}= P\bm{y}$  that transforms the
        quadratic form $\bm{x}^TA\bm{x}$  into a quadratic form
        $  \bm{y}^TD\bm{y} $  with no cross-product term ($x_1x_2$ ) \\
        col of P in the theorem are the principle axes of xTAx.\\
        y is the coordinate vec of x relative to the orthonormal basis
        of $\R^n$ given by these principle axes 
    \subsubsection{Geometry of Pinciple Axes}
        if $\bm{x}^TA\bm{x} = c$ where A is an invertible 2x2 symm, then
        it is either an ellipse, hyperbola, intersecting lines, a point,
        or no points.\\
        If A is a DMX, then graph is in standard position.\\
        nonDMX is rotated out of standard position.\\
        The principle axes (determined by Evec of A) means finding
        a new coord system wrt which the graph is in standard position.\\
        SEE EX5
    \subsubsection{Classifying Quadratic Forms}
        quadratic form Q is:\\
        a. positive definite if $Q(\bm{x}) > 0$ for all $\bm{x}\ne \bm{0}$ \\
        b. negative definite if $Q(\bm{x}) < 0$ for all $\bm{x}\ne \bm{0}$ \\
        c. indefinite if $Q(\bm{x})$  assumes both positive and negative vals \\
        Q is positive semidefinite if $\ge 0$ \\
        Q is negative semidefinite if $\le 0$  
    \subsubsection{QF and EV Theorem}
        let A be nxn symm. then QF $\bm{x}^TA\bm{x}$ is:\\
        a. +def IFF EV of A are all positive\\
        b. -def IFF EV of A are all negative \\
        c. indefinite IFF A has both positive and negative EV\\
        see diagram
    \subsubsection{QF MX}
        positive definite MX is a SMX for which the QF is +def \\
        ...
\subsection{Constrained Optimization}
    the requirement that $\bm{x}$  is  a unit vector:\\
    mag = 1, mag squared = 1, $\bm{x}^T\bm{x}= 1$ \\
    See EX1 for min/max of Q 
    \subsubsection{Optimization for SMX Theorem}
        let A be SMX, m is min, M is max EV of A.\\
        value of $\bm{x}^TA\bm{x}$  is M when x is a unit Evec 
        u corresponding to M.\\
        the value of $\bm{x}^TA\bm{x}$ is m when x is a unit Evec
        corresponding to m.
    \subsubsection{Maximum for SMX Theorem(SEE NEXT THEOREM)}
        a,$\lambda $,$\bm{u}$  as in previous thm,
        max val of $\bm{x}^TA\bm{x}$  is subject to:
        \begin{align*}
            \bm{x}^T\bm{x}= 1\,\,\,,\bm{x}^T\bm{u}_{1}= 0
        \end{align*}
        is the second greatest EV $\lambda _{2}$, and this max is attained
        when $\bm{x}$  is an Evec $\bm{u_{2}}$  corresponding to $\lambda _{2}$ 
    \subsubsection{General Maxmimum for SMX Theorem}
        let A be symm nxn w/ orthogonal diagonalization $A= PDP^{-1}$ 
        where the entries on the diag of D are arranged so that
        each EV is less than the previous entry, where col of P are
        corresponding unit Evec u. Then for $k= 2,\dots ,n$ , the
        maximum value of $\bm{x}^TAx$  subject to:
        \begin{align*}
             \bm{x}^T\bm{x}= 1\,\,\,,\bm{x}^T\bm{u}_{1}= 0,\dots 
             \,\,\,,\bm{x}^T\bm{u}_{k-1}= 0
        \end{align*}
        is the EV $\lambda _{k}$  and this max is at
        $\bm{x}= \bm{u}_{k}$ 




\subsection{APPLICATIONS to Image Processing and Stats}
    later in STATS and round2   
\end{document}
