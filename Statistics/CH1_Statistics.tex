\documentclass[12pt]{article}
\usepackage{amsmath}
\usepackage{amssymb}
\usepackage{fancyhdr}
%\usepackage{pgfplots}
\usepackage{graphicx}
\usepackage{geometry}
\usepackage{bm}


\newcommand{\R}{\mathbb{R}}
\newcommand{\C}{\mathbb{C}}

%\allowdisplaybreak
\begin{document}
\title{CH 1 What is Stats}
\author{mrevanisworking}
\maketitle

\subsection{Statistics Terminology}
    \subsubsection{Stats Def}
    "The objective of statistics is to make an inference about a population based on
    information contained in a simple form that population and to
    provide an associated measure of goodness for the inference."
    \subsubsection{Stats Goal}
        1. Identify population of interest (who studying). \\
        2. inferential objective (what want to know).\\
        3. How to collect data
\subsection{Graphical Methods}
    For Graphs: points of subdivision should be chosen so that measurements 
    cannot be on the point of division. 5-20 intervals is good.
    \par
    we should learn py while reading these next few textbooks. true.
    \par
    We use Numerical Descriptive Measures to describe data. There are measures of central
    tendency and measures of variation.
\subsection{Numerical Methods}
    \subsubsection{Mean}
        \begin{equation}
            \bar y= \frac{1}{n} \sum_{i=1}^{n} y_i
        \end{equation}
        where $\bar y$  is a mean of a sample n. The populatio mean is $ \mu  $
    \subsubsection{Variance}
        Variance: sum of square of differences between measurements
        and mean over (n-1)
        \begin{equation}
            s ^{2} = \frac{1}{n-1} \sum_{i=1}^{n}(y _{i} - \bar y) ^{2}
        \end{equation}
    \subsubsection{Standard Deviation}
        SD is the positive 2root of variance:
        \begin{equation}
            s=\sqrt{s^{2}}
        \end{equation}
        the corresponding population SD is denoted by:
        $ \sigma = \sqrt{\sigma ^2 }  $ \\
        used for a single set of measurements
    \subsubsection{Empirical Rule}
        For a dist of measurements that is approx normal, it follows that
        the interval with end points:
        $ \mu \pm \sigma  $ contains ~68\% of measurements\\
        $ \mu \pm 2\sigma  $ contains ~95\% of measurements \\
        $ \mu = 3\sigma  $ contains almost all of the measurements
\subsection{Inferences}
    Assumptions cannot be definite from small fractions of a population\\
    Probabilists assume they know the structure of the PoI
    and use theory to compute.\\
    Staticians use sample then probability.

\end{document}  
